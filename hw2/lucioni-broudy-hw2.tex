\documentclass[solution, letterpaper]{cs121}

\usepackage{tikz-qtree}
\usepackage{graphicx}

%% Please fill in your name and collaboration statement here.
%\newcommand{\studentName}{Renzo Lucioni and Daniel Broudy}
%\newcommand{\collaborationStatement}{I collaborated with...}
\newcommand{\solncolor}{red}
\begin{document}

\header{2}{March 1, 2013, at 12:00 PM}{}{}

%%%%%%%%%%%%%%%%%%%%%%%%%%%%%%%%%%%%%%%%%%%%%%%%%%%%
\problem{} 

\subproblem Bright-or-dark: at least 75\% of the pixels are on, or at least 75\% of the pixels are off. \\

A perceptron that recognizes bright-or-dark takes a vector of 9 inputs {\bf x} in which an on pixel is represented with a 1 and an off pixel is represented with a 0. The perceptron is defined by the set of weights {\textbf{\emph{w}}}

has threshold activation function

\begin{equation}
  g(s)=\begin{cases}
    +1, & \text{if $s \leq .25$ or $s \geq .75$}.\\
    -1, & \text{otherwise}.
  \end{cases}
\end{equation}

\subproblem Top-bright: a larger fraction of pixels is on in the top row than in the bottom two rows. \\



\subproblem Connected: the set of pixels that are on is connected. \\




%%%%%%%%%%%%%%%%%%%%%%%%%%%%%%%%%%%%%%%%%%%%%%%%%%%%
\problem{}

%%%%%%%%%%%%%%%%%%%%%%%%%%%%%%%%%%%%%%%%%%%%%%%%%%%%
\problem{}

%%%%%%%%%%%%%%%%%%%%%%%%%%%%%%%%%%%%%%%%%%%%%%%%%%%%
\problem{}



\end{document}



